\documentclass[11pt]{article}
\usepackage{geometry}                % See geometry.pdf to learn the layout options. There are lots.
\geometry{letterpaper}                   % ... or a4paper or a5paper or ... 
%\geometry{landscape}                % Activate for for rotated page geometry
%\usepackage[parfill]{parskip}    % Activate to begin paragraphs with an empty line rather than an indent
\usepackage{graphicx}
\usepackage{amssymb}
\usepackage{fullpage}
\usepackage{epstopdf}
\DeclareGraphicsRule{.tif}{png}{.png}{`convert #1 `dirname #1`/`basename #1 .tif`.png}
\usepackage[parfill]{parskip}

\usepackage[UTF8]{ctex}

\begin{document}

\subsection*{Example: The Diet Problem}

\begin{eqnarray*}
 \textrm{Sets} &  F & \textrm{set of food items} \\
               &  N  & \textrm{set of nutrients} \\
 \textrm{Parameters} & c_i, i\in F  & \textrm{cost of food $i$}  \\
& n_{ji}, j \in N, i \in F & \textrm{amount of nutrient $j$ in food $i$}  \\
& n^L_j, j\in N  &\textrm{minimum amount of nutrient $j$ required} \\
& n^U_j, j\in N  &\textrm{maximum amount of nutrient $j$ allowed} \\
 \textrm{Variables} & X_i, i\in F & \textrm{amount of food $i$ to consume}
\end{eqnarray*}
\begin{eqnarray*}
\textrm{Minimize:}\\
& \displaystyle \sum_{i\in F} c_i X_i & \textrm{minimize total cost of food}\\
\textrm{Subject to:}\\
& n^L_j \leq \sum_{i \in F} n_{ji} X_{i} \leq n^U_j \quad \forall j \in N & \textrm{consumption of nutrients in allowable range}
\end{eqnarray*}

This sample solution is concise; it gives the linear program
formulation with text describing each element of the model but nothing
more. Some explanations, and classes of the optimization problem. What specific graph problems does it solve?


\newpage
\subsection*{Problem 1: Shortest path problem}

\begin{eqnarray*}
    \textrm{Sets} \\
        & V & \textrm{set of vertices} \\
        & E & \textrm{set of edges} \\
    \textrm{Constants} \\
        & a, b \in V & \textrm{shortest path from $a$ to $b$} \\
    \textrm{Variables} \\
        & X \subseteq E & \textrm{selected edges} \\
    \textrm{Functions} \\
        & \begin{array}{rl}
            l(\{i, j\}) &= \left\{
                \begin{array}{ll}
                    0,                              & i == j \\
                    \infty,                         & \{i, j\} \notin E \\
                    \mathrm{length\ of\ }\{i, j\},  & \{i, j\} \in E \\
                \end{array}
            \right. \\
            l(i, j) &= l(\{i, j\})
        \end{array}, i, j \in V & \textrm{distance between vertices} \\
    \textrm{Minimize} \\
        & \displaystyle \sum_{x \in X} l(x) & \textrm{最小化路径长度} \\
    \textrm{Subject to} \\ 
        & \begin{array}{rl}
            \exists & \{i_1, j_1\}, \{i_2, j_2\}, ..., \{i_k, j_k\}, ..., \{i_n, j_n\} \\
            \mathrm{s.t.,} & i_1 = a, i_k = j_{k-1}, j_n = b \\
                    & \{i_k, j_k\} \in X, n \leq |X|, k = 2, 3, ..., n
        \end{array} & \textrm{保证为一条 $a$ 到 $b$ 的路径} \\ 
\end{eqnarray*}


\newpage
\subsection*{Problem 2: Disjoint-set problem}

\begin{eqnarray*}
    \textrm{Sets} \\
        & V & \textrm{set of vertices} \\
        & E & \textrm{set of edges} \\
    \textrm{Variables}
        & x_i, i \in V & \textrm{通常认为是点所属的连通块的序号} \\
    \textrm{Minimize} \\
        & |\{x_i | i \in V \}| & \textrm{最小化连通块数量} \\
    \textrm{Subject to} \\
        & c_i == c_j \to \{i, j\} \in E, i \neq j & \textrm{两点存在边才可标记为联通} \\
\end{eqnarray*}

当在无向图上,当连通块数量最小时,得到了一个并查集数据结构。当在有向图上,该问题为强联通分量问题。

\newpage
\subsection*{Problem 3: 最大环问题}

\begin{eqnarray*}
    \textrm{Sets} \\
        & V & \textrm{set of vertices} \\
        & E & \textrm{set of edges} \\
    \textrm{Variables} \\
        & X \subseteq E & \textrm{selected edges} \\
    \textrm{Functions} \\
        & \begin{array}{rl}
            l(\{i, j\}) &= \left\{
                \begin{array}{ll}
                    0,                              & i == j \\
                    \infty,                         & \{i, j\} \notin E \\
                    \mathrm{length\ of\ }\{i, j\},  & \{i, j\} \in E \\
                \end{array}
            \right. \\
            l(i, j) &= l(\{i, j\})
        \end{array}, i, j \in V & \textrm{distance between vertices} \\
    \textrm{Maximize} \\
        & \displaystyle \sum_{x \in X} l(x) & \textrm{最大化路径长度} \\
    \textrm{Subject to} \\
        & \begin{array}{rl}
            \exists & \{i_1, j_1\}, \{i_2, j_2\}, ..., \{i_k, j_k\}, ..., \{i_n, j_n\} \\
            \mathrm{s.t.,} & i_1 = j_n, i_k = j_{k-1} \\
                    & \{i_k, j_k\} \in X, n = |X|, k = 2, 3, ..., n
        \end{array} & \textrm{保证为一个环} \\ 
\end{eqnarray*}


\newpage
\subsection*{Problem 4: Travelling salesman problem}

\begin{eqnarray*}
    \textrm{Sets} \\
        & V & \textrm{set of vertices} \\
        & E & \textrm{set of edges} \\
    \textrm{Variables} \\
        & X \subseteq E & \textrm{selected edges} \\
    \textrm{Functions} \\
        & \begin{array}{rl}
            l(\{i, j\}) &= \left\{
                \begin{array}{ll}
                    0,                              & i == j \\
                    \infty,                         & \{i, j\} \notin E \\
                    \mathrm{length\ of\ }\{i, j\},  & \{i, j\} \in E \\
                \end{array}
            \right. \\
            l(i, j) &= l(\{i, j\})
        \end{array}, i, j \in V & \textrm{distance between vertices} \\
    \textrm{Minimize} \\
        & \displaystyle \sum_{x \in X} l(x) & \textrm{最小化路径长度} \\
    \textrm{Subject to} \\
        & |X| = |V| \\
        & \begin{array}{rl}
            \exists & \{i_1, j_1\}, \{i_2, j_2\}, ..., \{i_k, j_k\}, ..., \{i_n, j_n\} \\
            \mathrm{s.t.,} & i_1 = j_n, i_k = j_{k-1} \\
                    & \{i_k, j_k\} \in X, n = |X|, k = 2, 3, ..., n
        \end{array} & \textrm{保证为一个环} \\ 
\end{eqnarray*}


\newpage
\subsection*{Problem 5: 二分图匹配}

\begin{eqnarray*}
    \textrm{Sets} \\
        & V & \textrm{set of vertices} \\
        & E & \textrm{set of edges} \\
    \textrm{Variables} \\
        & X \subseteq E & \textrm{selected edges} \\
    \textrm{Minimize} \\
        & \displaystyle |X| & \textrm{最大化匹配数量} \\
    \textrm{Subject to} \\
        & |\{i | \{i, j\} \in X\} \cup \{j | \{i, j\} \in X\}| = 2 |X| & \textrm{保证没有重复匹配} \\ 
\end{eqnarray*}


\newpage
\subsection*{Problem 6: 最小支撑树问题}

\begin{eqnarray*}
    \textrm{Sets} \\
        & V & \textrm{set of vertices} \\
        & E & \textrm{set of edges} \\
    \textrm{Variables} \\
        & X \subseteq E & \textrm{selected edges} \\
    \textrm{Functions} \\
        & \begin{array}{rl}
            l(\{i, j\}) &= \left\{
                \begin{array}{ll}
                    0,                              & i == j \\
                    \infty,                         & \{i, j\} \notin E \\
                    \mathrm{length\ of\ }\{i, j\},  & \{i, j\} \in E \\
                \end{array}
            \right. \\
            l(i, j) &= l(\{i, j\})
        \end{array}, i, j \in V & \textrm{distance between vertices} \\
    \textrm{Constants} \\
        & A \subseteq V & \textrm{求 $A$ 的最小支撑树} \\
    \textrm{Minimize} \\
        & \displaystyle \sum_{x \in X} l(x) & \textrm{最小化支撑树路径长度} \\
    \textrm{Subject to} \\
        & \begin{array}{rl}
            \forall & a, b \in A, a \neq b \\
            \exists & \{i_1, j_1\}, \{i_2, j_2\}, ..., \{i_k, j_k\}, ..., \{i_n, j_n\} \\
            \mathrm{s.t.,} & i_1 = a, i_k = j_{k-1}, j_n = b \\
                    & \{i_k, j_k\} \in X, n \leq |X|, k = 2, 3, ..., n
        \end{array} & \textrm{保证有一条 $a$ 到 $b$ 的路径} \\ 
\end{eqnarray*}


\newpage
\subsection*{Problem 7: 最小单向支撑树问题}

\begin{eqnarray*}
    \textrm{Sets} \\
        & V & \textrm{set of vertices} \\
        & E & \textrm{set of edges} \\
    \textrm{Variables} \\
        & X \subseteq E & \textrm{selected edges} \\
    \textrm{Functions} \\
        & \begin{array}{rl}
            l(\{i, j\}) &= \left\{
                \begin{array}{ll}
                    0,                              & i == j \\
                    \infty,                         & \{i, j\} \notin E \\
                    \mathrm{length\ of\ }\{i, j\},  & \{i, j\} \in E \\
                \end{array}
            \right. \\
            l(i, j) &= l(\{i, j\})
        \end{array}, i, j \in V & \textrm{distance between vertices} \\
    \textrm{Constants} \\
        & a \subseteq V, B \in V & \textrm{求 $a$ 到 $B$ 的最小支撑树} \\
    \textrm{Minimize} \\
        & \displaystyle \sum_{x \in X} l(x) & \textrm{最小化支撑树路径长度} \\
    \textrm{Subject to} \\
        & \begin{array}{rl}
            \forall & b \in B \\
            \exists & \{i_1, j_1\}, \{i_2, j_2\}, ..., \{i_k, j_k\}, ..., \{i_n, j_n\} \\
            \mathrm{s.t.,} & i_1 = a, i_k = j_{k-1}, j_n = b \\
                    & \{i_k, j_k\} \in X, n \leq |X|, k = 2, 3, ..., n
        \end{array} & \textrm{保证有一条 $a$ 到 $b$ 的路径} \\ 
\end{eqnarray*}


\newpage
\subsection*{Problem 8: Maximum flow problem}

\begin{eqnarray*}
    \textrm{Sets} \\
        & V & \textrm{set of vertices} \\
        & E & \textrm{set of edges} \\
    \textrm{Constants} \\
        & s \in V & \textrm{源点} \\
        & t \in V & \textrm{汇点} \\
    \textrm{Functions} \\
        & \begin{array}{rl}
            w(\{i, j\}) &= \left\{
                \begin{array}{ll}
                    0, & i == j \\
                    0, & \{i, j\} \notin E \\
                    \mathrm{weight\ of\ }\{i, j\}, & \{i, j\} \in E \\
                \end{array}
            \right. \\
            w(i, j) &= w(\{i, j\})
        \end{array}, i, j \in V & \textrm{边权重} \\
    \textrm{Parameters} \\
        & f_{uv}, u, v \in V & \textrm{每条边的流量} \\
    \textrm{Maximize} \\
        & \displaystyle \sum_{v: \{s, v\} \in E} f_{sv} & \textrm{最大化流} \\
    \textrm{Subject to} \\
        & f_{uv} \leq w(u, v), \forall \{u, v\} \in E & \textrm{容量限制} \\
        & \displaystyle \sum_{v: \{u, v\} \in E} f_{uv} = \sum_{v: \{v, w\} \in E} f_{vw}, \forall v \in V - \{s, t\} & \textrm{流的保留} \\ 
\end{eqnarray*}


\newpage
\subsection*{Problem 9: Four color theorem}

\begin{eqnarray*}
    \textrm{Sets} \\
        & V & \textrm{set of vertices} \\
        & E & \textrm{set of edges} \\
    \textrm{Variables} \\
        & x_i, i \in V & \textrm{点为地图中的国家 $,x_i$ 为其颜色} \\
    \textrm{Minimize} \\
        & |\{x_i | i \in V\}| & \textrm{最小化颜色数量} \\
    \textrm{Subject to} \\
        & \{i, j\} \in E \to x_i \neq x_j & \textrm{相邻国家不同色} \\
\end{eqnarray*}

\newpage
\subsection*{Problem 10: 菊花图中心问题}

\begin{eqnarray*}
    \textrm{Sets} \\
        & V & \textrm{set of vertices} \\
        & E & \textrm{set of edges} \\
    \textrm{Variables} \\
        & x \in V & \textrm{菊花图中心} \\
    \textrm{Maximize} \\
        & |\{\{x, j\} | \{x, j\} \in E\}| & \textrm{最大化权重} \\
\end{eqnarray*}

\newpage
\subsection*{Problem 11: 双菊花图切割问题}

\begin{eqnarray*}
    \textrm{Sets} \\
        & V & \textrm{set of vertices} \\
        & E & \textrm{set of edges} \\
    \textrm{Constants} \\
        & a, b \in V & \textrm{两个菊花中心} \\
    \textrm{Variables} \\
        & x \in V & \textrm{切割点} \\
    \textrm{Functions} \\
        & \begin{array}{rl}
            f_1(\{i, j\}) &= \textrm{Problem 1 中的最短路} \\
            f_1(i, j) &= f_1(\{i, j\}) \\
        \end{array}, i, j \in V & \textrm{最短路} \\
    \textrm{Minimize} \\
        & f_1(a, x) + f_1(x, b) & \textrm{最小化最短路之和} \\
\end{eqnarray*}


\newpage
\subsection*{Problem 12: 最小费用最大流问题}

\begin{eqnarray*}
    \textrm{Sets} \\
        & V & \textrm{set of vertices} \\
        & E & \textrm{set of edges} \\
    \textrm{Constants} \\
        & s \in V & \textrm{源点} \\
        & t \in V & \textrm{汇点} \\
        & 0 \leq \beta < 1 & \textrm{费用权重参数} \\
    \textrm{Functions} \\
        & \begin{array}{rl}
            w(\{i, j\}) &= \left\{
                \begin{array}{ll}
                    0, & i == j \\
                    0, & \{i, j\} \notin E \\
                    \mathrm{weight\ of\ }\{i, j\}, & \{i, j\} \in E \\
                \end{array}
            \right. \\
            w(i, j) &= w(\{i, j\})
        \end{array}, i, j \in V & \textrm{边权重} \\
    \textrm{Parameters} \\
        & f_{uv}, u, v \in V & \textrm{每条边的流量} \\
    \textrm{Maximize} \\
        & \displaystyle (\sum_{v: \{s, v\} \in E} f_{sv}) - \beta \sum_{u, v: \{u, v\} \in E} f_{uv} & \textrm{} \\
    \textrm{Subject to} \\
        & f_{uv} \leq w(u, v), \forall \{u, v\} \in E & \textrm{容量限制} \\
        & \displaystyle \sum_{v: \{u, v\} \in E} f_{uv} = \sum_{v: \{v, w\} \in E} f_{vw}, \forall v \in V - \{s, t\} & \textrm{流的保留} \\ 
\end{eqnarray*}

在算法竞赛中,通常 $\beta$ 将非常小,使得项 $\displaystyle \beta \sum_{u, v: \{u, v\} \in E} f_{uv} < 1, 1$ 为单位流量.


\newpage
\subsection*{Problem 13: 最小生成树问题}

\begin{eqnarray*}
    \textrm{Sets} \\
        & V & \textrm{set of vertices} \\
        & E & \textrm{set of edges} \\
    \textrm{Variables} \\
        & X \subseteq E & \textrm{selected edges} \\
    \textrm{Functions} \\
        & \begin{array}{rl}
            l(\{i, j\}) &= \left\{
                \begin{array}{ll}
                    0,                              & i == j \\
                    \infty,                         & \{i, j\} \notin E \\
                    \mathrm{length\ of\ }\{i, j\},  & \{i, j\} \in E \\
                \end{array}
            \right. \\
            l(i, j) &= l(\{i, j\})
        \end{array}, i, j \in V & \textrm{distance between vertices} \\
    \textrm{Minimize} \\
        & \displaystyle \sum_{x \in X} l(x) & \textrm{最小化生成树路径长度} \\
    \textrm{Subject to} \\
        & |X| = |V| - 1 & \textrm{树的边数} \\
        & \begin{array}{rl}
            \forall & a, b \in V, a \neq b \\
            \exists & \{i_1, j_1\}, \{i_2, j_2\}, ..., \{i_k, j_k\}, ..., \{i_n, j_n\} \\
            \mathrm{s.t.,} & i_1 = a, i_k = j_{k-1}, j_n = b \\
                    & \{i_k, j_k\} \in X, n < |V|, k = 2, 3, ..., n
        \end{array} & \textrm{保证有一条 $a$ 到 $b$ 的路径} \\ 
\end{eqnarray*}


\newpage
\subsection*{Problem 14: 点划分问题}

\begin{eqnarray*}
    \textrm{Sets} \\
        & V & \textrm{set of vertices} \\
        & E & \textrm{set of edges} \\
    \textrm{Constants} \\
        & c & \textrm{点划分数量} \\
    \textrm{Variables} \\
        & \mathcal{X} & \textrm{点的分划} \\
    \textrm{Functions} \\
        & w(i) = \mathrm{weight\ of\ } i, i \in V & \textrm{点权重} \\
        & W(X) = \sum_{x \in X} w(x) , X \subseteq V & \textrm{点集权重} \\
    \textrm{Minimize} \\
        & \displaystyle \max_{X, Y \in \mathcal{X}} W(X) - W(Y) & \textrm{最小化差距} \\
    \textrm{Subject to} \\
        & |\mathcal{X}| = c & \\
        & \begin{array}{rl}
            \forall & X \in \mathcal{X} \\
            \forall & a, b \in X, a \neq b \\
            \exists & \{i_1, j_1\}, \{i_2, j_2\}, ..., \{i_k, j_k\}, ..., \{i_n, j_n\} \\
            \mathrm{s.t.,} & i_1 = a, i_k = j_{k-1}, j_n = b \\
                    & i_k, j_k \in X, \{i_k, j_k\} \in E, n < |V|, k = 2, 3, ..., n
        \end{array} & \textrm{保证 $X$ 构成联通图} \\ 
\end{eqnarray*}
\newpage


\subsection*{Problem 15: 边划分问题}

\begin{eqnarray*}
    \textrm{Sets} \\
        & V & \textrm{set of vertices} \\
        & E & \textrm{set of edges} \\
    \textrm{Constants} \\
        & c & \textrm{边划分数量} \\
    \textrm{Variables} \\
        & \mathcal{X} & \textrm{边的分划} \\
    \textrm{Functions} \\
        & \begin{array}{rl}
            w(\{i, j\}) &= \left\{
                \begin{array}{ll}
                    0, & i == j \\
                    0, & \{i, j\} \notin E \\
                    \mathrm{weight\ of\ }\{i, j\}, & \{i, j\} \in E \\
                \end{array}
            \right. \\
            w(i, j) &= w(\{i, j\})
        \end{array}, i, j \in V & \textrm{边权重} \\
        & \displaystyle W(X) = \sum_{x \in X} w(x) , X \subseteq V & \textrm{边集权重} \\
        & V(X) = \{i | \{i, j\} \in X\} & \textrm{边集涉及的点集} \\
    \textrm{Minimize} \\
        & \displaystyle \max_{X, Y \in \mathcal{X}} W(X) - W(Y) & \textrm{最小化差距} \\
    \textrm{Subject to} \\
        & |\mathcal{X}| = c & \\
        & \begin{array}{rl}
            \forall & X \in \mathcal{X} \\
            \forall & a, b \in V(X), a \neq b \\
            \exists & \{i_1, j_1\}, \{i_2, j_2\}, ..., \{i_k, j_k\}, ..., \{i_n, j_n\} \\
            \mathrm{s.t.,} & i_1 = a, i_k = j_{k-1}, j_n = b \\
                    & \{i_k, j_k\} \in X, n < |V|, k = 2, 3, ..., n
        \end{array} & \textrm{保证 $X$ 构成联通图} \\ 
\end{eqnarray*}


\newpage
\subsection*{Problem 16: 最大团问题}

\begin{eqnarray*}
    \textrm{Sets} \\
        & V & \textrm{set of vertices} \\
        & E & \textrm{set of edges} \\
    \textrm{Variables} \\
        & X \subseteq V & \textrm{selected vertices} \\
    \textrm{Maximize} \\
        & \displaystyle |X| & \textrm{最大化团大小} \\
    \textrm{Subject to} \\
        & \{a, b\} \in E, \forall a, b \in V & \textrm{保证是团} \\ 
\end{eqnarray*}


\newpage
\subsection*{Problem 17: 最小割问题}

\begin{eqnarray*}
    \textrm{Sets} \\
        & V & \textrm{set of vertices} \\
        & E & \textrm{set of edges} \\
    \textrm{Constants} \\
        & s \in V & \textrm{源点} \\
        & t \in V & \textrm{汇点} \\
    \textrm{Variables} \\
        & X \subseteq E & \textrm{selected edges} \\
    \textrm{Functions} \\
        & \begin{array}{rl}
            l(\{i, j\}) &= \left\{
                \begin{array}{ll}
                    0,                              & i == j \\
                    \infty,                         & \{i, j\} \notin E \\
                    \mathrm{length\ of\ }\{i, j\},  & \{i, j\} \in E \\
                \end{array}
            \right. \\
            l(i, j) &= l(\{i, j\})
        \end{array}, i, j \in V & \textrm{distance between vertices} \\
    \textrm{Minimize} \\
        & \displaystyle \sum_{x \in X} l(x) & \textrm{最小化代价} \\
    \textrm{Subject to} \\
        & \begin{array}{rl}
            \exists & A, B \subset V \\
            \mathrm{s.t.,} & s \in A, t \in B, \{a, b\} \notin E - X, \forall a \in A, b \in B \\
        \end{array} & \textrm{保证 s 和 t 不联通} \\ 
\end{eqnarray*}


\newpage
\subsection*{Problem 18: 树的最优根结点问题}

\begin{eqnarray*}
    \textrm{Sets} \\
        & V & \textrm{set of vertices} \\
        & E & \textrm{set of edges} \\
    \textrm{Variables} \\
        & x \in V & \textrm{根结点} \\
    \textrm{Functions} \\
        & \begin{array}{rl}
            f_1(\{i, j\}) &= \textrm{Problem 1 中的最短路} \\
            f_1(i, j) &= f_1(\{i, j\}) \\
        \end{array}, i, j \in V & \textrm{最短路} \\
    \textrm{Minimize} \\
        & \displaystyle \sum_{v \in V} f_1(x, v) & \textrm{最小化最短路之和} \\
\end{eqnarray*}


\newpage
\subsection*{Problem 19: 最大生成链问题}

\begin{eqnarray*}
    \textrm{Sets} \\
        & V & \textrm{set of vertices} \\
        & E & \textrm{set of edges} \\
    \textrm{Variables} \\
        & X \subseteq E & \textrm{selected edges} \\
    \textrm{Functions} \\
        & \begin{array}{rl}
            l(\{i, j\}) &= \left\{
                \begin{array}{ll}
                    0,                              & i == j \\
                    \infty,                         & \{i, j\} \notin E \\
                    \mathrm{length\ of\ }\{i, j\},  & \{i, j\} \in E \\
                \end{array}
            \right. \\
            l(i, j) &= l(\{i, j\})
        \end{array}, i, j \in V & \textrm{distance between vertices} \\
        & V(X) = \{i, j | \{i, j\} \in X\} & \textrm{边集涉及的点集} \\
    \textrm{Maximize} \\
        & \displaystyle \sum_{x \in X} l(x) & \textrm{最大化路径长度} \\
    \textrm{Subject to} \\ 
        & |X| = |V(X)| - 1 & \textrm{链也是树} \\
        & \begin{array}{rl}
            \exists & \{i_1, j_1\}, \{i_2, j_2\}, ..., \{i_k, j_k\}, ..., \{i_n, j_n\} \\
            \mathrm{s.t.,} & i_1 \neq j_n, i_k = j_{k-1}\\
                    & \{i_k, j_k\} \in X, n \leq |X|, k = 2, 3, ..., n
        \end{array} & \textrm{保证为一条路径} \\ 
\end{eqnarray*}


\newpage
\subsection*{Problem 20: 边染色问题}

\begin{eqnarray*}
    \textrm{Sets} \\
        & V & \textrm{set of vertices} \\
        & E & \textrm{set of edges} \\
    \textrm{Variables} \\
        & x_{ij}, \{i, j\} \in E & \textrm{$x_i$ 为边的颜色} \\
    \textrm{Minimize} \\
        & |\{x_i | i \in V\}| & \textrm{最小化颜色数量} \\
    \textrm{Subject to} \\
        & \{i, j\}, \{i, k\} \in E \to x_{ij} \neq x_{ik} & \textrm{相邻边不同色} \\
\end{eqnarray*}



\newpage
\subsection*{Problem 21: 独立集问题}

\begin{eqnarray*}
    \textrm{Sets} \\
        & V & \textrm{set of vertices} \\
        & E & \textrm{set of edges} \\
    \textrm{Variables} \\
        & X \subseteq V & \textrm{selected vertices} \\
    \textrm{Maximize} \\
        & \displaystyle |X| & \textrm{最大化独立集大小} \\
    \textrm{Subject to} \\
        & \{a, b\} \notin E, \forall a, b \in V & \textrm{保证是独立集} \\ 
\end{eqnarray*}


\newpage
\subsection*{Problem 22: 路线检查问题}

\begin{eqnarray*}
    \textrm{Sets} \\
        & V & \textrm{set of vertices} \\
        & E & \textrm{set of edges} \\
    \textrm{Variables} \\
        & x_1, x_2, ..., x_n \subseteq E & \textrm{是一个边序列} \\
    \textrm{Functions} \\
        & \begin{array}{rl}
            l(\{i, j\}) &= \left\{
                \begin{array}{ll}
                    0,                              & i == j \\
                    \infty,                         & \{i, j\} \notin E \\
                    \mathrm{length\ of\ }\{i, j\},  & \{i, j\} \in E \\
                \end{array}
            \right. \\
            l(i, j) &= l(\{i, j\})
        \end{array}, i, j \in V & \textrm{distance between vertices} \\
    \textrm{Minimize} \\
        & \displaystyle \sum_{i=1}^n l(x_i) & \textrm{最小化路径长度} \\
    \textrm{Subject to} \\
        & n \in \mathbb{N}^+ & \\
        & \begin{array}{rl}
            \forall & e \in E, \\
            \exists & i \in \mathbb{N} \cap [1, n] \\
            \mathrm{s.t.,} & x_i = e \\
        \end{array} & \textrm{保证为一个环} \\ 
\end{eqnarray*}


\newpage
\subsection*{Problem 23: 最小支配集问题}

\begin{eqnarray*}
    \textrm{Sets} \\
        & V & \textrm{set of vertices} \\
        & E & \textrm{set of edges} \\
    \textrm{Variables} \\
        & X \subseteq V & \textrm{selected vertices} \\
    \textrm{Functions} \\
        & \begin{array}{rl}
            l(\{i, j\}) &= \left\{
                \begin{array}{ll}
                    0,                              & i == j \\
                    \infty,                         & \{i, j\} \notin E \\
                    \mathrm{length\ of\ }\{i, j\},  & \{i, j\} \in E \\
                \end{array}
            \right. \\
            l(i, j) &= l(\{i, j\})
        \end{array}, i, j \in V & \textrm{distance between vertices} \\
    \textrm{Minimize} \\
        & \displaystyle |X| & \textrm{最小化支配集大小} \\
    \textrm{Subject to} \\
        & \begin{array}{rl}
            \forall & v \in V \\
            \exists & x \in X \\
            \mathrm{s.t.,} & l(x, v) \neq \infty \\
        \end{array} & \textrm{保证为一个支配集} \\ 
\end{eqnarray*}


\newpage
\subsection*{Problem 24: 斯坦纳树问题}

\begin{eqnarray*}
    \textrm{Sets} \\
        & V & \textrm{set of vertices} \\
        & E & \textrm{set of edges} \\
    \textrm{Variables} \\
        & X \subseteq E & \textrm{selected edges} \\
    \textrm{Functions} \\
        & \begin{array}{rl}
            l(\{i, j\}) &= \left\{
                \begin{array}{ll}
                    0,                              & i == j \\
                    \infty,                         & \{i, j\} \notin E \\
                    \mathrm{length\ of\ }\{i, j\},  & \{i, j\} \in E \\
                \end{array}
            \right. \\
            l(i, j) &= l(\{i, j\})
        \end{array}, i, j \in V & \textrm{distance between vertices} \\
    \textrm{Constants} \\
        & a \in V & \textrm{求 $a$ 的斯坦纳树} \\
    \textrm{Minimize} \\
        & \displaystyle \sum_{x \in X} l(x) & \textrm{最小化斯坦纳树路径长度} \\
    \textrm{Subject to} \\
        & |X| = |V| - 1 & \textrm{树的边数} \\
        & \begin{array}{rl}
            \forall & b \in V, b \neq a \\
            \exists & \{i_1, j_1\}, \{i_2, j_2\}, ..., \{i_k, j_k\}, ..., \{i_n, j_n\} \\
            \mathrm{s.t.,} & i_1 = a, i_k = j_{k-1}, j_n = b \\
                    & \{i_k, j_k\} \in X, n \leq |X|, k = 2, 3, ..., n
        \end{array} & \textrm{保证有一条 $a$ 到 $b$ 的路径} \\ 
\end{eqnarray*}


\newpage
\subsection*{Problem 25: 最大联通块问题}

\begin{eqnarray*}
    \textrm{Sets} \\
        & V & \textrm{set of vertices} \\
        & E & \textrm{set of edges} \\
    \textrm{Variables} \\
        & X \subseteq V & \textrm{selected vertices} \\
    \textrm{Maximize} \\
        & \displaystyle |X| & \textrm{最大化点集} \\
    \textrm{Subject to} \\ 
        & \begin{array}{rl}
            \forall & a, b \in X, \\
            \exists & \{i_1, j_1\}, \{i_2, j_2\}, ..., \{i_k, j_k\}, ..., \{i_n, j_n\} \\
            \mathrm{s.t.,} & i_1 = a, i_k = j_{k-1}, j_n = b \\
                    & \{i_k, j_k\} \in X, n \leq |X|, k = 2, 3, ..., n
        \end{array} & \textrm{保证有一条 $a$ 到 $b$ 的路径} \\ 
\end{eqnarray*}

\end{document}  
